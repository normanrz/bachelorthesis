\chapter{Conclusion}

\section{Summary}
This Bachelor Thesis covered the architecture of the client applicaition of Project Zoom. Project Zoom is an IT-System that was developed for the HPI School of Design Thinking (D-School) as part of the Bachelor Project G1 2012 at the Hasso Plattner Institute.

First, the use cases and requirements, which were a result of the design process, were outlined. Afterwards, the design of Project Zoom was explained. Its user interface offers three main Views that address the needs of different stakeholders. Users navigate from one View to the next by zooming in or out. This concept is called Semantic Zoom. The application is based on a client-server architecture for it to be also accessible outside of the D-School. Project Zoom relies on web technologies to achieve support for multiple device platforms including mobile tablet devices. Client and server communicate via a REST interface. The client application is designed to use the Model-View-Controller (MVC) pattern, which enforces the separation between presentation and business logic. The application uses an event-driven approach for communication between the system's components. Furthermore, it has been discussed that Operational Transformation is a well-fitting technique to provide a real-time collaboration experience. 

Then, the components of the client application have been introduced. The Model is the component that is responsible for connecting to the server's REST interface. It stores the data in a hierarchical structure and keeps track of any changes. It also listens to changes sent from the server and assembles patches for change propagation to the server. The View is responsible for rendering the data from the Model to the user interface. It is built with a set of hierarchical components that bind to the data. The Controller handles user interactions and manipulates both the data in the Model as well as the View. There is a main Controller for handling the zoom-based actions and several Behavior Controllers that encapsulate behavior for a particular View. 

After outlining the architecture concepts, some implementation considerations have been discussed. Finally, the architecture has been evaluated based on the previously established use cases and requirements.

\section{Future Work}
Project Zoom supports real-time collaboration by implementing the Operational Transformation concept.  However, due to the lack of proper conflict resolution, only concurrent read access is encouraged. Developing a solution for handling conflicts could enable the full potential of a real-time collaborative application.

The current implementation of Project Zoom has support for mobile tablet devices in a way that all main tasks can be completed. However, improving performance and taking advantage of the devices' multi-touch gestures could enhance the user experience greatly.

