\pdfbookmark[0]{Zusammenfassung}{abstract_de}
\chapter*{Zusammenfassung}
\hypertarget{abstract_de}{}

Project Zoom ist ein Software Tool, dass für die HPI School of Design Thinking (D-School) entwickelt wurde, um die Dokumentation von innovativen Projekten zu unterstützen. Beim Arbeiten an Design Thinking Projekten, erzeugen die Studententeams sehr viele Dokumente, wie z.B. Präsentationen, Fotos, Videos und Textdokumente. Das vorgestellte Tool erlaubt den Teams ihren Kreativprozess in einem interaktiven Graphen zu visualisieren. Das ermöglicht nicht nur den Teammitgliedern, sondern auch Teachers und Mitarbeiter, wichtige Erkenntnisse auf den Projekten zu gewinnen. Diese Informationen werden persistent gespeichert, sodass sie auch noch Monate nach Fertigstellung des Projektes verfügbar sind. Außerdem bietet das Tool eine Übersicht über alle Projekte der D-School, was die Akquise von neuen Projekten erleichtert.

Diese Bachelorarbeit beschreibt die Clientanwendung von Project Zoom. Zuerst werden die Use Cases und Anforderungen an das System, die im Entwicklungsprozess identifiziert wurden, vorgestellt. Project Zoom wurde mit einer web-basierten Client-Server Architektur entworfen. Damit können viele Computer-Plattformen unterstützt werden. Außerdem, kann die Software über das Internet aufgerufen werden. Die Clientanwendung verwendet eine Ereignis-gesteuerte Architektur und implementiert das Konzept ,,Operational Transformation'', um Kollaboration in Echtzeit zu ermöglichen. Da die Anwendung auf dem Model-View-Controller (MVC) Entwurfsmuster aufbaut, werden auch die Model-, View- und Controller-Komponenten detailliert beschrieben. Abschließend, werden einige Merkmale der Implementierung des Systems erläutert und das System anhand der Use Cases und Anforderungen evaluiert.