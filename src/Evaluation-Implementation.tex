\chapter{Evaluation and Implementation}

\section{Use cases and requirements revisited}

This section explains how the proposed architecture of Project Zoom enables the use cases and fulfills the requirements, that were previously established.

\begin{figure}
\begin{center}
\missingfigure{Use case 1 figure}
\caption{Sequence diagram for use case 1.}
\label{fig:evaluc1}
\end{center}
\end{figure}

\paragraph{\ref{uc:organize}} Figure \ref{fig:evaluc1} shows how files are pulled from a storage provider by the server\footnote{Werkmeister covers how the data is pulled from different kinds of storage providers, e.g. \textsc{Box} and \textsc{Filemaker}. \cite{Werkmeister_2013}} and sent to the client. Because of the implemented dragging behavior, a user is able to add these documents as a node to the visual graph. Using context-sensitive actions these node may be connected to others or annotated in different ways \cite{Herold_2013}. 

\begin{figure}
\begin{center}
\missingfigure{Use case 2 figure}
\caption{Sequence diagram for use case 2.}
\label{fig:evaluc2}
\end{center}
\end{figure}

\paragraph{\ref{uc:display}} The D-School uses a \textsc{Filemaker}\footnote{Filemaker, \url{http://www.filemaker.com/}, accessed 06/16/13} database to store projects. The server polls that databases. The client can then access them through a REST interface and display them in a user interface. The interaction is outlined in figure \ref{evaluc2}.

\paragraph{\ref{uc:fromhome}} Project Zoom is designed with a web-based client-server architecture. The server software is capable of handling requests from clients over a network connection. If the software is deployed on a public server, the application will be reachable from any internet-connected computer. Use case \ref{uc:fromhome} requires the users to have an installation of a HTML5-capable browser to access the application. Such web browsers are very popular applications and are very likely to be already preinstalled.

\paragraph{\ref{uc:versions}} When a graph in Project Zoom has been edited, it is automatically stored shortly afterwards in the server's database. Every version of a particular graph is kept in the database for later retrieval \cite{Bocklisch_2013}. The server's REST interface includes a method fro retrieving specific versions of a graph (see: Appedix \ref{appendix:REST}).

\paragraph{\ref{uc:storageproviders}} D-School students store the files they create in their projects using cloud storage providers, e.g. \textsc{Box}. The server includes a connector to the \textsc{Box}-API. The extensible architecture of the server also allows to integrate with other services \cite{Werkmeister_2013}. Because of the event-driven architecture of the client, added files are displayed after a very short time (see: figure \ref{fig:evaluc1}).

\begin{figure}
\begin{center}
\missingfigure{Use case 6 figure}
\caption{Photo of Project Zoom running on an Apple iPad (3rd generation).}
\label{fig:evaluc6}
\end{center}
\end{figure}

\paragraph{\ref{uc:multiplatform}} The D-School runs a variety of students projects. Using the \textsc{Filemaker} database, Project Zoom is able to display them in the user interface. Users are able to select a filter, which only shows the matching items \cite{Dieckhoff_2013}. Because the client of Project Zoom is implemented using HTML5 technologies, it also runs on popular mobile tablet devices, as demonstrated in figure \ref{fig:evaluc6}.

\paragraph{\ref{req:multiplatform}, \ref{req:fromhome}, \ref{req:storageprovider}} The requirement \textbf{\ref{req:multiplatform}} has already been covered by \ref{uc:multiplatform}, \textbf{\ref{req:fromhome}} by \ref{uc:fromhome} and \textbf{\ref{req:storageprovider}} by \ref{uc:storageproviders}.

\paragraph{\ref{req:concurrency}} Due to the nature of a web application, multiple clients can connect to a server. Thus, mulitple users can access the same application concurrently. The system imposes no restrictions on concurrent read access to any resource through the client's user interface. An Operational Transformation algorithm is applied to provide a realtime collaboration experience and ensure eventual consistency. Because of the very basic handling of consistency conflicts, concurrent editing of the same resource (e.g. a graph) is not supported in the current implementation and subject to future work.

\section{Implementation considerations}

This section documents some of the implementation problems that have been solved while developing the client application of Project Zoom. Some of the concepts introduced here are important to know when extending the system.

\subsection{Module and file management}
Modularization is a common pattern in software development. Modules are pieces of code that have a particular responsibility within the system. They provide a well-defined interface for other modules to consume and explicitly list their dependencies. This decoupling leads to better maintainability, as the code is encapsulated and modules may easily be replaced. \cite{Osmani_2011}

JavaScript in its current version does not support modules natively. A common approach is to separate the code into different files and have them loaded through \texttt{<script>} tags in the DOM. This technique is prone to naming conflicts. Because the files share the same global scope, variables are shared as well. This can be avoided by wrapping a file's in an immediate function \cite{Resig_2013}. However, to export their functionality the files usually append properties to the global object, i.e. \texttt{window}. For larger systems this leads to \textit{namespace pollution}. Another issue is that the files have no means of declaring their dependencies programmatically. Thus, the ordering of the \texttt{<script>} tags is significant.

The Asynchronous Module Definition (AMD) format\footnote{Asynchronous Module Definition, \url{https://github.com/amdjs/amdjs-api/wiki/AMD}, accessed 06/27/13} provides a module implementation for JavaScript. Modules are defined by specifying a function and a list of dependencies. In conjunction with an AMD script loader, these modules can be loaded asynchronously. The script loader ensures that the respective dependencies are resolved in advance. Also, there are tools for concatenating the module files.\footnote{Concatenating scripts into one file is a common technique to reduce page-loading time by decreasing the required HTTP requests.} The AMD format has been used to organize the code of the client application.

\subsection{Testability}
Automatic tests are a popular technique for improving code quality. 

* Modules with very small set of dependencies: Units
* Mocking dependencies (Network access, User interface), by exchanging AMD module resolvers
* Test suite of Project Zoom for client Model

\subsection{Memory leaks in JavaScript}
JavaScript is a managed-memory language. Runtime implementations include a garbage collector (GC) that frees unnecessary objects. To determine the state of an object, its references in the system are examined. If an object has no references that are reachable from a particular set of root objects then it is considered unnecessary. Because there are higher order function in JavaScript and scopes are realized through closures\footnote{Closures are execution contexts that allow a function to access variables that are extern to its definition. \cite{Resig_2013}} removing all references of an object may become a tedious task in larger systems. 

A popular solution to this problem is to employ an event dispatcher. An event dispatcher is a singleton\footnote{A singleton is an instance of a class that is guaranteed to only have one instance. \cite{Gamma_1994}} that keeps track of all callbacks including event handlers as well as their sender and receiver objects. Thus, removing all callbacks related to an object is reduced to a single method invocation. This solution is implemented in the client's Model component.


\subsection{Event cycles}

\begin{figure}
\begin{center}
\missingfigure{Event cycle example}
\caption{An example of an infinite loop in the event-system.}
\label{fig:eventcycle}
\end{center}
\end{figure}

When using an event-based system it is possible to create infinite execution loop. An example is shown in figure \ref{fig:eventcycle}. A common solution to that problem is to keep track of the object that initiated an event and omit event handlers of that object while propagating the event. \todo{citation missing}

\subsection{Event congestion}
As shown in figure \ref{fig:eventcycle} changes to objects of the Model get triggered by native browser events, such as \texttt{mousemove}. Their frequency is determined by the sample rate of the input device (e.g. mouse or trackpad) and may be as high as 60 signals per second. This is a favorable effect, as it enables a lag-free user interface. However, as the event-driven architecture is designed to propagate change events not only within the client application but also to the server, this high event rate may lead to congestion of the network connection. 
A solution to this problem is to throttle the network requests based on a fixed time interval. With this technique states in which the client has not yet completed his action, e.g. not released the mouse and still dragging, are still propagated to the server. Addressing this issue, there is another approach that only sends requests after there have been no events for a fixed time span. This mechanism, which is called \textit{debouncing}, is used for the client's synchronization with the server.
