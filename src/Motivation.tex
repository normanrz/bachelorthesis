\chapter{Motivation}

This Bachelor thesis documents a part of the Bachelor project G1 2012 at the Hasso Plattner Institute in Potsdam. A group of six students developed a documentation tool for the Hasso Plattner Institute School of Design Thinking (D-School) \footnote{\url{http://www.hpi.uni-potsdam.de/d_school/home.html?L=1}}. 

\section{Design Thinking at Hasso Plattner Institute}
The School of Design Thinking offers academic courses for students. Design Thinking is a method for creating new ideas and developing novel solutions \cite{Plattner_2009}. 

During the courses the students work on team projects. Starting in the Basic Track there are 1-week, 3-week and 6-week projects. In the Advanced Track students work 12 weeks continuously on one project. In addition to the students tracks the D-School also offers Executive Training where the projects usually don't exceed one week in length. 

As one of its core principles the D-School actively encourages multidisciplinary teams. This mixture of different background leads to multiple viewpoints during the design phases and helps the team to filter out obstacles early in the process. 

The projects usually deal with a problem posed by an industry partner. The Design Thinking method includes an iterative process which consists of several well-defined phases. The phases guide the teams from basic understandings to the development of testable prototypes. The process is non-linear because the teams are encouraged to iterate. These cycles help to refine the prototypes based on actual user feedback. Throughout the process the students are being advised by teachers. \missingfigure{Design Thinking phase bubbles}

The D-School provides several team workspaces. A typical workspace consists of a set of whiteboards and a standing desk, which is large enough for the whole team to work on. The D-School considers open spaces to be important for the creative process, so the equipment is usually movable. Therefore, the spaces are hosted in large halls. The D-School also provides tools and materials for rapid prototyping.

There are several staff members at the D-School for coordinating and acquiring the student's projects. Key stakeholders for this thesis are the Head of D-School, the Program Manager, the Knowledge Manager and the Track Managers.

\section{Design process}
The Bachelor project also applied an iterative process, which is very similar to the one taught at the D-School. With this approach the project participants were able to benefit from the Design Thinking method while automatically learning about some of the needs of the D-School students.

During the project there have been several interviews with the relevant stakeholders. The group learned about the student's documentation efforts, especially commonly used methods and tools. Other stakeholders like teachers or D-School staff members have also been interviewed. Based on this information prototypes were developed to enhance the documenting experience of the students. These prototypes were evaluated during user testing sessions.

\section{Project Zoom}
In the second iteration a new prototype was designed which also addressed the need of the staff members to archive and categorize the projects for easy retrieval. This prototype is called ``Project Zoom". The following sections will highlight the most relevant use cases and requirements for this thesis \footnote{The ``Software Requirements Specification"\cite{ReqSpec} covers the use cases and requirements in detail}.

\subsection{Use Cases}
Project Zoom addresses two main use cases, which were distilled from the insights gathered in the observation phases.

\begin{enumerate}
\item Student teams document their projects by organizing the digital documents they created in a visual manner.\\
\textsc{Prerequisite}: The students have the documents digitally available.\\
\textsc{Postcondition}: A visual knowledge graph is being created.

\item D-School staff members get an overview of all projects. This overview enables access to the projects' classifications, related people and documentations.\\
\textsc{Prerequisite}: The projects were entered in a database and the students have documented their projects.\\
\textsc{Postcondition}: A visual representation of the projects is displayed.
\end{enumerate}

Beyond these two major use cases there are other relevant use cases.

\begin{enumerate}
\setcounter{enumi}{2}

\item The Head of D-School and Program Manager show an overview of a set of projects to potential industry partners using a mobile tablet device.\\
\textsc{Prerequisite}: The projects were entered in a database.\\
\textsc{Postcondition}: A visual representation of the projects is displayed.

\item The students add additional documents from their computers at home.\\
\textsc{Prerequisite}: The students have the documents digitally available.\\
\textsc{Postcondition}: The documents are included in the knowledge graph.

\item Students, teachers and D-School staff members can retrieve states of the knowledge graph at different points in time.\\
\textsc{Prerequisite}: The students have documented their projects using the proposed tool.\\
\textsc{Postcondition}: A historical version of the knowledge graph is displayed.

\item Students can access their documents through different commonly-used storage services, e.g. \textsc{Box}\footnote{The D-School uses \textsc{Box} as a shared storage service. \url{https://www.box.com/}}.\\
\textsc{Prerequisite}: The students stored documents using the respective service.\\
\textsc{Postcondition}: Documents are placed in the knowledge graph.
\end{enumerate}

\subsection{Technical Requirements}

To fulfill these use cases there are some technical requirements for designing an implementation of Project Zoom.

\begin{enumerate}
\item The system's interface has to support multiple platforms, including the popular desktop operating systems and modern tablet devices. (use cases 4-5)
\item The system's interface has to be accessible from outside the D-School. (use case 4) 
\item The systems has to support concurrent users accessing and modifying contents. \textit{Simplifying assumption}: Only one user is modifying one ressource (e.g. a project's graph) at the same time. (general requirement)
\item The system connects to different data sources, e.g. \textsc{Box}, and makes the stored data available through its GUI. (use case 1,6)
\item The system connects to the database the D-School uses to store projects and contacts. (use case 2)

\end{enumerate}

\subsection{Design}

Project Zoom is designed to apply the concept of semantic zooming. Semantic Zoom is part of the computer interface model \textsc{Pad} proposed by Perlin and Fox \cite{Perlin_1993}. This concept taps into the natural spatial thinking of users. Information is displayed on a large infinite two-dimensional canvas. Users browse around either by panning or zooming. There are different representations of the same pieces of information at different magnification levels. When zoomed out documents are only be represented by an icon or a title string. While zooming in these abstract figures gradually resolve into the full representation of the respective documents. 

In Project Zoom there are three main zoom levels: An overview of all projects (Overview view), a detailed view of a project and its metadata (Details view), and a view hosting a project's knowledge graph (Process view). Users can navigate from one view to the next by zooming in and out. This approach allows building one cohesive system, which hosts interfaces that are targeted at the specific needs of different stakeholders (see use cases 1-6).

The main views themselves also support different zoom levels. Especially the Process view takes advantage of the zooming capabilities and displays the documents in the graph in different levels of detail. 
The Process view is an interactive graph that allows users to add documents onto the canvas and connect them with edges. The document nodes can then be annotate using the commenting feature or by drawing clusters (e.g. free form shapes) around them. These actions are exposed through context-sensitive menus. Also, the interactive graph features automatic layout capabilities, like collision prevention.


\subsection{Choosing a basic architecture}
In the design phase multiple architectures for Project Zoom have been evaluated. 

A \textbf{monolithic application} \footnote{Wikipedia, Monolithic application, \url{http://en.wikipedia.org/w/index.php?title=Monolithic_application&oldid=552899667}} is a self-contained system that can run on a single computer. The benefits are that the data is consistent for all users.  Also, the system is easy to set up because there is only one computer required for the system to work. This approach is well suited for applications that deal with independent datasets, which can be stored on one platform. Word processors are an example of monolithic applications. However, as use cases 4 and 5 as well as requirements 1 and 3 equire the data to be accessible on multiple platforms concurrently this architectural approach isn't suitable for Project Zoom.

The \textbf{peer-to-peer} \cite{Schollmeier_2001} architecture allows the distribution of data on multiple connected computers. Each node in the network is equally priviledged and handles a subset of the data individually. Data consistency can be eventually achieved through replication from on neighbor node to another. This is a well-known difficult database problem \cite{Gray_1996}. Access control can be integrated by using trusted computing technology \cite{Sandhu_2005}. Peer-to-peer architectures are usually applied when a centralized controlling instance is to be avoid. Bitcoin is a popular peer-to-peer application. The ability of a node to function, especially when joining the network, depends on the availability of neighbor nodes. Because of the low number of users the availability of nodes might pose as an issue. Also, implementing a peer-to-peer usually requires to build a native application which adds complexity to fulfilling requirement 1.

The \textbf{client-server} \cite{Berson_1996} model is one of the most-used distributed architecture. One central computer acts as a server which stores the dataset and manages access control. Client computers can access and modify the resources on the server through a network interface. An example of a client-server application is the World Wide Web. Fundamentally, the server is a single point of failure \footnote{Wikipedia, Single point of failure, \url{http://en.wikipedia.org/w/index.php?title=Single_point_of_failure&oldid=555725127}}. However, there are methods to maximizing the availability \cite{Gray_1991} \cite{Colyer_2000}. This architecture is suitable for Project Zoom as it fulfills all the requirements, especially no. 1-3, and allows all the use cases.

A web-based client-server architecture is a system that uses an HTTP-Server \cite{RFC2616} and a browser application as client. Project Zoom incorporates this approach for several reasons:
\begin{itemize}
\item Using the client application of a web-based system only requires the installation of a web browser, which are wide-spread and usually preinstalled in the operating system. This makes it easy for students to modify the data from their computers at home (see: use case 4).
\item Developing native apps for mobile devices requires different technologies for each platform \cite{Charland_2011}. However, recent mobile devices are equipped with HTML5-capable \footnote{HTML 5 is what?} web browsers. Applications built using web technologies are likely to run on multiple mobile platforms (see: requirement 1).
\item Because of the standardization of the web technologies, emerging platforms are likely to support them aswell.
\end{itemize}

Alternatives to a web-based client-server architecture are based on client applications that users have to install on their devices. Because the users of Project Zoom use multiple computing devices, this isn't a favorable approach, as the application has to be installed on each device seperately. Also, users may not even be able to install the application on to the device because of imposed restrictions.

\subsection{Architecture overview}

\begin{figure}
\missingfigure{Complete Architecture diagram}
\caption{Architecture overview of Project Zoom}
\label{fig:CompleteArchitectureDiagram}
\end{figure}

Figure \ref{fig:CompleteArchitectureDiagram} shows the system architecture of Project Zoom. The server implementation features an event-system that is fed by the data connectors and sets up the pipeline for thumbnail generation. The server also hosts the database model and handles user access control and authentication. Client and server are connected through an HTTP-REST interface \cite{Fielding_2000}. The client application is built using an MVC pattern \footnote{Addy Osmani, \url{http://addyosmani.com/resources/essentialjsdesignpatterns/book/\#detailmvcmvp} \cite{Osmani_2012}}, which later chapters will cover in detail. 

\subsection{Related work}
There are six theses covering Project Zoom. Bocklisch \cite{Bocklisch_2013} describes the architecture of the server including the data model. Werkmeister \cite{Werkmeister_2013} details the connection of the data providers, e.g. \textsc{Box} and \textsc{Filemaker}, to the system. Bräunlein's thesis \cite{Braeunlein_2013} covers both the design and technical generation of document thumbnails for the Process View. Dieckhoff \cite{Dieckhoff_2013} details the automatic layout capabilities of the interactive graphs. Herold \cite{Herold_2013} explains the context-sensitive actions that users can perform on the graph and its nodes and edges. This thesis is about the architecture of the client application. 